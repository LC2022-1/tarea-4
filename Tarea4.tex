\documentclass{article}

% Símbolos
\usepackage{recycle}
\usepackage{amsmath}
\usepackage{amsfonts}
\usepackage{amssymb}

% Figuras
\usepackage{graphicx}

% Márgenes
\addtolength{\voffset}{-1cm}
\addtolength{\hoffset}{-1.5cm}
\addtolength{\textwidth}{3cm}
\addtolength{\textheight}{2cm}

%% Definimos lema.
\newtheorem{lemma}{Lema}

% Encabezados y Pies de Página
\usepackage{fancyhdr}
% Información del Encabezado
\lhead{L\'ogica Computacional 2022-1 \\
       Tarea 4}
     \rhead{Profesor: V\'ictor Zamora Guti\'errez \\
       Ayudantes: Jos\'e Ricardo Desales Santos\\
  Alma Rosa P\'aes Alcal\'a}
% Información del Pie de Página
\rfoot{\recycle}
\cfoot{\vspace{-0.8cm}?`Realmente necesitas imprimir esta hoja?}
\lfoot{\recycle}
\pagenumbering{gobble}
% Estilo
\pagestyle{fancyplain}

\begin{document}

\begin{enumerate}
  
\item Decimos que un lenguaje es reconocible si existe un
  algoritmo que, dada una cadena, nos responda ``s\'i'' cuando la
  cadena pertenece al lenguaje (el algoritmo no da ninguna garant\'ia cuando la
  cadena no pertenece al lenguaje).
  \begin{enumerate}
  \item ?`Qu\'e significa que un lenguaje sea reconocible pero no decidible?
  \item D\'e un ejemplo de un lenguaje reconocible pero no decidible (distinto
    al del inciso \ref{itm:haskell}).
  \item \label{itm:haskell} Demuestre que el lenguaje de los programas en Haskell que se detienen es reconocible.
  \end{enumerate}

\item Modifique el algoritmo de tableaux para que, dada una f\'ormula como
  entrada, responda ``s\'i'' cuando la f\'ormula es v\'alida y ``no'' cuando la
  f\'ormula es insatisfacible.\\
  \textbf{Hint:} Investigar la t\'ecnica de \textit{dovetailing}.
  
\end{enumerate}

\end{document}
  